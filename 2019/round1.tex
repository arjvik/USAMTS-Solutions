\documentclass[12pt]{article}
\usepackage{amsmath,amssymb,amsthm}
\usepackage[pdftex]{graphicx}
\usepackage[inline]{asymptote}
\usepackage{fancyhdr}
\usepackage[english]{babel}
\pagestyle{fancy}
\usepackage[nottoc]{tocbibind}
\usepackage{hyperref}
\usepackage{setspace}
\usepackage{csquotes}
\usepackage{titlesec}
\usepackage{float}
\usepackage{wasysym}
\usepackage{enumitem}
\include{usamts}
\renewcommand{\baselinestretch}{1.44}
\setlength{\parindent}{0cm}

\realname{Arjun Vikram}
\username{arjvik}
\usamtsid{36783}
\usamtsyear{31}
\usamtsround{1}

\begin{document}

%% Problem 1 Solution
\begin{solution}{1}
    \begin{asy}
        unitsize(1cm);
        for(int i = 0; i < 10; ++i) {
            for(int j = 0; j < 10; ++j) {
                draw((i - 0.5, j - 0.5)--(i + 0.5, j - 0.5)--(i + 0.5, j + 0.5)--(i - 0.5, j + 0.5)--(i - 0.5, j - 0.5), gray(0.5));
            }
        }
        
        draw((0 - 0.5, 2 - 0.5)--(1 - 0.5, 2 - 0.5), gray(0) + 3);
        draw((0 - 0.5, 6 - 0.5)--(0.5, 5.5), black+3);
        draw((-0.5, 6.5)--(0.5, 6.5)--(0.5, 7.5), black+3);
        draw((-0.5, 8.5)--(0.5, 8.5), black+3);
        draw((1.5, -0.5)--(1.5, 1.5), black+3);
        draw((1.5, 5.5)--(1.5, 6.5)--(2.5, 6.5)--(2.5, 7.5)--(1.5, 7.5), black+3);
        draw((1.5, 9.5)--(1.5, 8.5), black+3);
        draw((2.5, -0.5)--(2.5, 0.5)--(3.5, 0.5)--(3.5, 1.5), black+3);
        draw((4.5, 1.5)--(5.5, 1.5)--(5.5, 2.5), black+3);
        draw((4.5, 6.5)--(5.5, 6.5), black+3);
        draw((5.5, 8.5)--(6.5, 8.5)--(6.5, 7.5), black+3);
        draw((6.5, -0.5)--(6.5, 0.5), black+3);
        draw((6.5, 1.5)--(7.5, 1.5), black+3);
        draw((8.5, 5.5)--(8.5, 4.5), black+3);
        
        // Set up utilities for solution
        
        pen c;
        c = black+1;
        
        void horizontal(int x, int y) {
            draw((x, y)--(x+1,y), c);
        }
        void vertical(int x, int y) {
            draw((x, y)--(x,y+1), c);
        }
        void star(int x, int y) {
            label("$\large\bigstar$", (x, y), 0*dir(0), c);
        }
        
        // Solution starts here
        
        horizontal(0, 9);
        vertical(0, 7);
        vertical(2, 8);
        horizontal(1, 7);
        star(1, 8);
        horizontal(0, 6);
        vertical(2, 5);
        star(1, 5);
        star(2, 4);
        vertical(0, 4);
        vertical(1, 3);
        vertical(2, 2);
        vertical(0, 2);
        star(1, 2);
        horizontal(0, 0);
        horizontal(0, 1);
        star(1,2);
        vertical(2, 0);
        vertical(3, 1);
        star(3, 3);
        vertical(3, 4);
        star(3, 6);
        vertical(3, 7);
        star(3, 9);
        horizontal(3, 0);
        horizontal(5, 0);
        horizontal(4, 1);
        star(6, 1);
        star(4, 2);
        vertical(5, 2);
        star(5, 4);
        vertical(4, 3);
        star(4, 5);
        horizontal(6, 2);
        horizontal(7, 3);
        horizontal(6, 4);
        horizontal(5, 5);
        horizontal(4, 6);
        star(6, 3);
        star(4, 7);
        star(6, 6);
        vertical(5, 7);
        vertical(6, 7);
        vertical(4, 8);
        star(5, 9);
        horizontal(6, 9);
        horizontal(8, 9);
        horizontal(7, 8);
        star(9, 8);
        star(7, 7);
        horizontal(8, 7);
        horizontal(7, 6);
        star(9, 6);
        vertical(9, 4);
        star(9, 3);
        horizontal(7, 5);
        star(8, 4);
        vertical(7, 0);
        vertical(8, 0);
        vertical(9, 1);
        star(8, 2);
        star(9, 0);
    \end{asy}
\end{solution}

%% Problem 2 S on
\begin{solution}{2}

    We are given $x,y,z\in\mathbb R$, $x,y,z > 1$, and
    \begin{equation*}
        x^y=y^z=z^x.
    \end{equation*}
    
    WLOG, let us assume that $x \le y \le z$.
    
    Because $\ln x$ is monotonically increasing, $\ln x \le \ln y \le \ln z$.
    
    We can take the natural log of both sides of the given equation to get
    \begin{align*}
        \ln x^y &= \ln y^z = \ln z^x \\
        y\ln x &= z\ln y = x\ln z. \tag{1}
    \end{align*}
    
    Note that we were able to do this because $x, y, z > 1$.
    
    We can then use \textbf{Lemma 1} (proved below) with the following substitutions:
    \begin{equation*}
        a = y \qquad
        b = z \qquad
        c = \ln x \qquad
        d = \ln y.
    \end{equation*}
    
    By Lemma 1,
    \begin{equation*}
        y\ln x + z\ln y \ge y\ln y + z\ln x. \tag{2}
    \end{equation*}
    
    We now try to use AM-GM on the right hand side of the equation.
    \begin{align*}
        \frac{y\ln y + z\ln x}2
        &\ge \sqrt{yz\cdot\ln x\cdot\ln y} \\
        &= \sqrt{y\ln x\cdot z\ln y} \\
        &= \sqrt{y\ln x\cdot y\ln x} \qquad\small\text{(using equation 1)} \\
        &= \sqrt{\left(y\ln x\right)^2} \\
        &= y\ln x.
    \end{align*}
    
    Then
    \begin{align*}
        y\ln y + z\ln x
        &\ge 2y\ln x \\
        &= y\ln x + y\ln x \\
        &= y\ln x + z\ln y. \tag{3}
    \end{align*}
    
    Combining inequalities $2$ and $3$
    \begin{equation*}
        y\ln x + z\ln y \ge y\ln y + z\ln x \ge y\ln x + z\ln y. \tag{4}
    \end{equation*}
    
    Notice that the left and right sides of inequality $(4)$ are the same expression.
    Therefore, since the expression in the middle of the inequality is sandwiched between the left and right sides,
    \begin{equation*}
        y\ln x + z\ln y = y\ln y + z\ln x.
    \end{equation*}

    By \textbf{Lemma 1 Corollary} (proved below), either $y=z$ or $\ln x = \ln y \implies x=y$. We can prove $x=y=z$ pretty easily in both cases as follows.

    \textbf{Case 1:} $y=z$
    \begin{align*}
        y^z &= z^x \tag{\small{from given}} \\
        z^z &= z^x \\
        \ln{z^z} &= \ln{z^x} \\
        z\ln z &= x\ln z\\
        z &= x,
    \end{align*}
    
    so therefore,
    \begin{equation*}
        x=y=z. \tag*{\qed}
    \end{equation*}

    \pagebreak
    \textbf{Case 2:} $x=y$

    This case is quite similar to Case 1.
    
    \begin{align*}
        x^y &= y^z \tag{\small{from given}} \\
        y^y &= y^z \\
        \ln{y^y} &= \ln{y^z} \\
        y\ln y &= z\ln y \\
        y &= z,
    \end{align*}
    
    so therefore,
    \begin{equation*}
        x=y=z. \tag*{\qed}
    \end{equation*}
    
    \subsection*{Lemma 1}
    This lemma is the two-variable case of the Rearrangement Inequality.
    
    Let
    \begin{align*}
        a &\le b, \\
        c &\le d.
    \end{align*}
    
    Then,
    \begin{align*}
        b-a &\ge 0 \\
        d-c &\ge 0,
    \end{align*}
    
    so
    \begin{equation*}
        (b-a)(d-c) \ge 0. \tag{A}
    \end{equation*}
    
    Expanding this,
    \begin{equation*}
        bd+ac - ad - bc \ge 0.
    \end{equation*}
    
    If we move two of the terms to the right hand side,
    \begin{equation*}
        ac + bd \ge ad + bc. \tag*{\qed}
    \end{equation*}
    
    \subsubsection*{Lemma 1 Corollary}
    (Equality case)
    
    Now assume
    \begin{equation*}
        ac+bd=ad+bc.
    \end{equation*}
    
    We can see that this equality can only be reached if either
    \begin{equation*}
        b-a = 0
        \text{ or }
        d-c = 0,
        \tag{\small{see inequality A}}
    \end{equation*}
    
    so either $a=b$ or $c=d$. Therefore, we have proved that for $a\le b,\ c\le d$,
    \begin{equation*}
        ac+bd=ad+bc \implies a=b \text{ or } c=d. \tag*{\qed}
    \end{equation*}
    
\end{solution}

\begin{solution}{3}

    \begin{asydef}
        unitsize(5cm);
        
        pair p = (1,1);
        pair l = (0,1);
        pair u = (0,0);
        pair m = (1,0);
        
        pen text = black+1;
        pen line = black+1;
        pen thin = black+.5;
        pen dash = line+dashed;
        pen point= black+3;
        
        draw(p, point);
        draw(l, point);
        draw(u, point);
        draw(m, point);
        
        draw(p--l, line);
        draw(l--u, line);
        draw(u--m, line);
        draw(m--p, line);
        
        label("$P$", p, 1*dir(45), text);
        label("$L$", l, 1*dir(135), text);
        label("$U$", u, 1*dir(225), text);
        label("$M$", m, 1*dir(315), text);
        
        pair w = (.5, .5);
        /*draw(w, point);
        label("$\omega$", w, 1*dir(135), text);*/
        draw(circle(w, .5), line);
    \end{asydef}
    
    \begin{figure}[H]
        \centering
        \begin{asy}
            draw((0.5,-0.1)--(0.5,1.1), thin);
            draw((-0.1,0.5)--(1.1,0.5), thin);
            label("I", (0.7, 0.7), text);
            label("II", (0.3, 0.7), text);
            label("III", (0.3, 0.3), text);
            label("IV", (0.7, 0.3), text);
            draw(w, point);
            label("$\omega$", w, 1*dir(135), text);
        \end{asy}
        \caption{Separating square $PLUM$ into quadrants}
    \end{figure}
    
    
    
    We begin by looking at the possible positions of the point $E$.
    Point $E$ could be in any of the 4 quadrants (I, II, III, IV).
    We can see that the case where point $E$ is in quadrant IV is
    just a reflection of the case where point $E$ is in quadrant II, so we can ignore that case.
    We can also see that if point $E$ is in quadrant III, point $I$ must be in either quadrant I, II, or IV (see Figure 2 below),
    so we can switch point $E$ and point $I$ and have point $E$ in either quadrant I, II, or IV.
    
    \begin{figure}[H]
        \centering
        \begin{asy}
            draw((0.5,-0.1)--(0.5,1.1), thin);
            draw((-0.1,0.5)--(1.1,0.5), thin);
            label("I", (0.7, 0.7), line);
            label("II", (0.3, 0.7), line);
            label("III", (0.3, 0.3), line);
            label("IV", (0.7, 0.3), line);
            
            pair e1 = (0.078, 0.2326);
            pair i1 = (0.322, 0.9674);
            draw(e1, point);
            draw(i1, point);
            label("$E_1$", e1, 1*dir(0), text);
            label("$I_1$", i1, 1*dir(315), text);
            draw(u--e1--i1, dash);
            
            pair e2 = (0.1464, 0.1464);
            pair i2 = (0.8536, 0.8536);
            draw(e2, point);
            draw(i2, point);
            label("$E_2$", e2, 1*dir(0), text);
            label("$I_2$", i2, 1*dir(270), text);
            draw(u--e2--i2, dash);
            
            pair e3 = (0.2326, 0.078);
            pair i3 = (0.9674, 0.322);
            draw(e3, point);
            draw(i3, point);
            label("$E_3$", e3, 3*dir(350), text);
            label("$I_3$", i3, 1*dir(135), text);
            draw(u--e3--i3, dash);
        \end{asy}
        \caption{Cases where pont $E$ is in quadrant III}
    \end{figure}
    
    Therefore, we only have to consider the cases where point $E$ is in quadrants I or II, as the other two cases can be reduced to these cases. We will now place square $PLUM$ onto the coordinate plane, where $U = (0, 0)$ and $P=(1, 1)$. 
    
    \begin{asydef}
        label("$(1,1)$", p, 4*dir(18), text);
        label("$(0,1)$", l, 4*dir(162), text);
        label("$(0,0)$", u, 4*dir(198), text);
        label("$(1,0)$", m, 4.2*dir(342), text);
        
    \end{asydef}
    
    \begin{figure}[H]
        \centering
        \begin{asy}
            draw(u--(0,1.2), line, EndArrow);
            draw(u--(1.4,0), line, EndArrow);
            draw(w, point);
            label("$\omega (\frac12, \frac12)$", w, 1*dir(0), text);
        \end{asy}
        \caption{Placing square $PLUM$ onto the coordinate plane}
    \end{figure}
    
    Let point $E = (a, b)$. We can let $a$ be a parameter, and define $b$ in terms of $a$. Because we limited point $E$ to being in quadrant I or II, there is a unique value $b$ for each value $a$.
    
    The equation of the circle $\omega$ is
    \begin{equation*}
        y = \frac12 \pm \sqrt{\frac14 - \left(x-\frac12\right)^2}.
    \end{equation*}
    
    Because we know that point $E$ is in quadrants I or II, we want to use the upper portion, so
    \begin{equation*}
        b = \frac12 + \sqrt{\frac14 - \left(a-\frac12\right)^2}. \tag{1}
    \end{equation*}
    
    Let point $I = (c, d)$. To find point $I$, we draw line $\overline{UE}$ and find where it intersects with circle $\omega$.
    
    \begin{figure}[H]
        \centering
        \begin{asy}
            pair e = (0.322, 0.964);
            draw(e, point);
            label("$E (a,b)$", e, 1*dir(315), text);
            
            pair i = (0.078, 0.2326);
            draw(i, point);
            label("$I (c,d)$", i, 1*dir(0), text);
            
            draw(u--i--e, line);
        \end{asy}
        \caption{Placing square $PLUM$ onto the coordinate plane}
    \end{figure}
    
    The equation of this line is
    \begin{equation*}
        y = \frac{bx}a.
    \end{equation*}
    
    Because point $I$ is in the lower half of the circle, we use the lower portion of our equation for circle $\omega$. Therefore, we can find the coordinates of point $I$ with the equation
    \begin{equation*}
        d = \frac{bc}a = \frac12 - \sqrt{\frac14 - \left(c-\frac12\right)^2}. \tag{2}
    \end{equation*}
    
    We can then solve this equation for $b$ and $c$ in terms of $a$ as follows:
    \begin{align*}
        &\frac{bc}a = \frac12 - \sqrt{\frac14 - \left(c-\frac12\right)^2} \\
        &\frac12 - \frac{bc}a = \sqrt{\frac14 - \left(c-\frac12\right)^2} \\
        &\left(\frac12 - \frac{bc}a\right)^2 = \frac14 - \left(c-\frac12\right)^2 \\
        &\frac14 + \frac{b^2c^2}{a^2} -\frac{bc}a = \frac14 - c^2 - \frac14 + c \\
        &\frac{b^2}{a^2}c^2 - \frac ba a + \frac14 = -c^2 + c \\
        &\left(\frac{b^2}{a^2} +1\right)c^2 - \left(\frac ba +1\right)c + \frac14 = 0. \\
    \end{align*}
    
    Using the quadratic formula, we see that we have solutions at
    \begin{align*}
        c
        &= \dfrac{\left(\dfrac ba +1\right)\pm\sqrt{
            \left(\dfrac ba +1\right)^2 - \left(\dfrac{b^2}{a^2} +1\right)
        }}{2\left(\dfrac{b^2}{a^2} +1\right)} \\
        &= \dfrac{\left(\dfrac ba +1\right)\pm\sqrt{
            \dfrac{b^2}{a^2} + 1 + \frac{2b}a - \dfrac{b^2}{a^2} - 1
        }}{2\left(\dfrac{b^2}{a^2} +1\right)} \\
        &= \dfrac{\left(\dfrac ba +1\right)\pm\sqrt{
            \dfrac{2b}a
        }}{2\left(\dfrac{b^2}{a^2} +1\right)}
    \end{align*}
    
    Clearly, we have an extraneous solution, as there is only one possible point $I$ for any given point $E$. If we plug in equation $(1)$ into the larger solution, after some heavy simplification (using the online math tool Wolfram Alpha) we get $c=a$. Therefore, we only need to consider the smaller solution, so
    \begin{equation*}
        c = \dfrac{
            \dfrac ba + 1 - \sqrt{\dfrac{2b}a}
        }{
            2\left(\dfrac{b^2}{a^2}+1\right)
        }. \tag{3}
    \end{equation*}
    
    Since we now have solved for $b$, $c$, and $d$ in terms of $a$, we are ready to derive an expression for the area
    of $\triangle PIE$.
    
    The coordinates of the vertices of the triangle are (in counter-clockwise order) $P=(1,1)$, $E=(a,b)$, and $I=(c,d)$.
    Using the Shoelace Theorem, the area of this triangle is:
    \begin{align*}
        \left[\triangle PIE\right]
        &= \left|\frac{ 1 \cdot b + a \cdot d + c \cdot 1 - 1 \cdot a - b \cdot c - d \cdot 1 }2\right| \\
        &= \left|\frac{ b + c - a - d + ad - bc }2\right|.
    \end{align*}
    
    Note that by rearranging equation $(2)$, $ad=bc$, so this becomes
    \begin{equation*}
        \left[\triangle PIE\right] = \left|\frac{b + c - a - d}2\right|. \tag{4}
    \end{equation*}
    
    Now, let us plug in our values for $b$, $c$, and $d$ into equation $(4)$, using equations $(1)$, $(3)$, and $(2)$ respectively.
    
    \begin{align*}
        \left[\triangle PIE\right]
        &= \frac12\big| b + c - d - a\big| \\
        &=\frac{1}{2}\left|\frac{1}{2}+\sqrt{\frac{1}{4}-\left(a-\frac{1}{2}\right)^{2}}+\frac{\frac{\frac{1}{2}+\sqrt{\frac{1}{4}-\left(a-\frac{1}{2}\right)^{2}}}{a}+1-\sqrt{\frac{2\left(\frac{1}{2}+\sqrt{\frac{1}{4}-\left(a-\frac{1}{2}\right)^{2}}\right)}{a}}}{2\left(\frac{\left(\frac{1}{2}+\sqrt{\frac{1}{4}-\left(a-\frac{1}{2}\right)^{2}}\right)^{2}}{a^{2}}+1\right)}\right.\\&\qquad\qquad\left.-
        \frac{1}{a}\cdot\frac{\frac{\frac{1}{2}+\sqrt{\frac{1}{4}-\left(a-\frac{1}{2}\right)^{2}}}{a}+1-\sqrt{\frac{2\left(\frac{1}{2}+\sqrt{\frac{1}{4}-\left(a-\frac{1}{2}\right)^{2}}\right)}{a}}}{2\left(\frac{\left(\frac{1}{2}+\sqrt{\frac{1}{4}-\left(a-\frac{1}{2}\right)^{2}}\right)^{2}}{a^{2}}+1\right)}\cdot\left(\frac{1}{2}+\sqrt{\frac{1}{4}-\left(a-\frac{1}{2}\right)^{2}}\right)-a\right|
    \end{align*}
    
    We now need to find the maximum of this expression. We do so by taking its derivative and setting it equal to 0.
    \begin{equation*}
        \frac{\mathrm d}{\mathrm da} [\triangle PIE] = 0
    \end{equation*}
    
    Unfortunately, this expression is too complex to differentiate by hand. We can make use of Wolfram Alpha to take it's derivative, and then set it equal to 0. This gives us a solution of $a = 0.25$. If we plug this solution back into the equation we have for $[\triangle PIE]$ (again using Wolfram Alpha to simplify), we get that
    \begin{equation*}
        \boxed{\max\ [\triangle PIE] = \frac14.} \tag*{\qed}
    \end{equation*}

\end{solution}

\begin{solution}{4}
    \textbf{NOTE: This is not intended to be a full solution. I was unable to figure out how to prove my conjectures. However, I would like to attempt to earn a few points from what I have discovered, if at all possible.}
    
    \newtheorem{conj}{Conjecture}
    \begin{conj}
        The number of people stuck with mangoes which can not be shared or eaten in a group of
        $K$ people with $N$ mangoes initially is the number of 1s in the binary representation of $N$,
        as long as $K$ is not smaller than this number.
    \end{conj}
    
    I unfortunately do not know how to prove this conjecture fully. What I have discovered is:
    
    \setlist{nolistsep}
    \begin{itemize}
        \itemsep 0em
        \item Someone is only stuck with mangoes that can't be shared or eaten if they have exactly one mango.
        \item The conjecture is true for various values of $N$ and $K$ when $1\le N\le 7$.
        \item I believe the final proof will be via induction on the different values of $K$ or $N$.
    \end{itemize}
    
    \paragraph{}
    This conjecture was discovered through the observation that if "sharing" was not allowed, then the number of people
    with a single mango at the end which can not be eaten is exactly the number of 1s when $N$ is written in binary.
    The proof is trivial: consider a person performing the "eating" operation as many times as possible. If they
    began with $N$ mangoes, they will pass $\left\lfloor\frac N2\right\rfloor$ mangoes to the next person. They will remain
    with either 0 or 1 mangoes if $N$ is even or odd, respectively. if $N$ was odd, the right-most digit in the binary representation of $N$ was a 1, and otherwise it was a 0. The number of mangoes received by the next person is then
    $N$ with its last digit removed. Thus, for every time someone is left with a mango, a $1$ is removed from the binary
    representation of the number of mangoes at the current person, until the next person gets no mangoes, so the number
    of people with mangoes is the number of $1$s in the binary representation of the number of mangoes at the beginning.
    \newline
    Using \textbf{Conjecture 1} ($K=100$, $N=2019$), there will be exactly 8 people left with mangoes.
    
\end{solution}

\begin{solution}{5}
    This proof is left as an exercise to the reader. \smiley
\end{solution}

\end{document}